\documentclass[12pt,letterpaper]{article}
\usepackage[utf8]{inputenc}
\usepackage[spanish]{babel}
\usepackage{graphicx}
\usepackage[left=2cm,right=2cm,top=2cm,bottom=2cm]{geometry}
\usepackage{graphicx} % figuras
% \usepackage{subfigure} % subfiguras
\usepackage{float} % para usar [H]
\usepackage{amsmath}
%\usepackage{txfonts}
\usepackage{stackrel} 
\usepackage{multirow}
\usepackage{enumerate} % enumerados
\renewcommand{\labelitemi}{$-$}
\renewcommand{\labelitemii}{$\cdot$}
% \author{}
% \title{Caratula}
\begin{document}

% Fancy Header and Footer
% \usepackage{fancyhdr}
% \pagestyle{fancy}
% \cfoot{}
% \rfoot{\thepage}
%

% \usepackage[hidelinks]{hyperref} % CREA HYPERVINCULOS EN INDICE

% \author{}
\title{Caratula}

\begin{titlepage}
\begin{center}
\large{UNIVERSIDAD PRIVADA DE TACNA}\\
\vspace*{-0.025in}
\begin{figure}[htb]
\begin{center}
\includegraphics[width=8cm]{./Imagenes/logo}
\end{center}
\end{figure}
\vspace*{0.15in}
INGENIERIA DE SISTEMAS  \\

\vspace*{0.5in}
\begin{large}
TITULO:\\
\end{large}

\vspace*{0.1in}
\begin{Large}
\textbf{SISTEMA CONTROL DE VERSIONES EN LINEA} \\
\end{Large}

\vspace*{0.3in}
\begin{Large}
\textbf{CURSO:} \\
\end{Large}

\vspace*{0.1in}
\begin{large}
BASE DE DATOS II\\
\end{large}

\vspace*{0.3in}
\begin{Large}
\textbf{DOCENTE:} \\
\end{Large}

\vspace*{0.1in}
\begin{large}
 Ing. Patrick Cuadros Quiroga\\
\end{large}

\vspace*{0.2in}
\vspace*{0.1in}
\begin{large}
Alumno: \\
\begin{flushleft}
Jhordy Vizcarra Llanque	           \hfill	(2015052719) \\
\end{flushleft}
\end{large}

\vspace*{0.3in}
\begin{Large}
\textbf{TACNA-PERU:} \\
\end{Large}

\vspace*{0.1in}
\begin{large}
2018\\
\end{large}

\end{center}

\end{titlepage}


\section{Sistema Control de Versiones}

Es una herramienta capaz de registrar todos los cambios que se realizan en uno o más proyectos, guardando a su vez versiones anteriores del proyecto, versiones a las que podemos acudir en caso de no funcionar de la forma correcta.


\section{GIT}
Es un sistema de control de versiones distribuido, cuyo principal objetivo es ayudar en el desarrollo de cualquier tipo de aplicación manteniendo una gran cantidad de código de un gran número de programadores diferentes. 

Esta herramienta fue impulsada por Linux Torvalds y el equipo de desarrollo del Kernel de Linux, que al igual que este sistema operativo, lo lanzaron como código abierto, además de encontrárnoslo para las distintas plataformas del mercado. Para hacer uso de ella, lo único que debemos hacer es instalar en nuestro sistema la versión correspondiente al sistema operativo que estemos utilizando. 

Aportando:

\begin{itemize}
    \item Auditoría completa del código, sabiendo en todo momento quién ha tocado algo, cuándo y qué. 
    \item Control sobre cómo ha ido cambiando nuestro proyecto con el paso del tiempo.  específico.
    \item Volver uno o más pasos hacia atrás de forma rápida. 
    \item Control de versiones del proyecto por medio de etiquetas.
    \item Seguridad, ya que todas las estructuras internas de datos irán cifradas con el algoritmo SHA1.
\end{itemize}


\section{GitHub}
Es un servidor donde alojar los repositorios de los proyectos, añadiendo funcionalidades extra para la gestión del proyecto y del código fuente. A diferencia del proyecto Git, éste se trata de un servicio comercial, ya que, aunque tiene una parte pública gratuita, cuenta con la desventaja de que todo el código que subamos estará disponible para cualquier persona. Si queremos decidir quién puede tener acceso a nuestro repositorio, entonces sería necesario pasarse a la modalidad de pago. 

Características que nos ofrece:

\begin{itemize}
    \item Una Wiki que opera con Git para el mantenimiento de las distintas versiones de las páginas.
    \item Sistema de seguimiento de problemas. Se trata de un sistema muy parecido al tradicional ticket, donde cualquier miembro del equipo o persona (si nuestro repositorio es público) puede abrir una consulta o sugerencia que se quiera hacer.
    \item Herramienta de revisión de código. Permite añadir anotaciones en cualquier punto de un fichero. 
    \item Visor de ramas que permite comparar los progresos realizados en las distintas ramas de nuestro repositorio.
\end{itemize}

\section{GITEA}
Gitea es una bifurcación gestionada por la comunidad de Gogs, el servicio de Gitea de Git auto hospedado es totalmente gratuito. Esta es una solución de alojamiento de código liviano escrita en Go y publicada bajo la licencia MIT.
También es un método simple y rápido de configurar un servicio de Git auto hospedado para el desarrollo de software de código abierto.



\section{BITBUCKET}
Bitbucket es un servicio de alojamiento basado en web, para los proyectos que utilizan el sistema de control de versiones Mercurial y Git. Bitbucket ofrece planes comerciales y gratuitos.
Bitbucket cuenta con bastantes características notables tales como la búsqueda de código, las solicitudes de extracción, cuenta con modelos de implementación flexibles, visualización de diferencias, la creación de reflejos inteligente, el seguimiento de problemas se puede configurar en el listas blancas de IP y permisos de sucursales para salvaguardar su flujo de trabajo.


\section{GitLab}
Es un repositorio de gestión de proyectos dotado de interfaz web. Como podemos deducir del nombre, está construido sobre Git, y básicamente nos proporciona el código para generar un servidor y gestionar los clientes, sus opciones y los servicios afrecidos.
A través de GitLab, podemos gestionar grupos, personas y los permisos que queremos que tengan los usuarios dentro de los grupos o proyectos a los que pertenezcan. También nos permite llevar a cabo un seguimiento del estado actual y del histórico de los proyectos pudiendo, así, ver todos los cambios y modificaciones producidas en el tiempo de desarrollo, además de gráficos, otros datos de interés de los proyectos y servicios que van más allá del control de versiones. Ejemplos de estos servicios serían los comentarios de usuarios sobre un proyecto, herramientas de planificación, issues (utilizados para reportar o avisar de errores), requests (para facilitar a la comunidad de proyectos compartidos, se permite que la gente haga peticiones de actualización con su código y que, si al propietario del proyecto le parece adecuado, puedan aceptarse). 

Que nos ofrece:

\begin{itemize}
    \item Opción de autentificar contra servicios como LDAP; Un punto interesante, ya que otros servicios similares a GitLab no nos ofrecían esta opción de autentificación.
    \item Distintos tipos de acceso y permisos (uso de roles y grupos); Restringiendo proyectos a ciertos usuarios y permitiendoles acceder a ciertos contenidos limitados o realizar ciertas acciones concretas. Los usuarios pueden acceder al protyecto a través de la web y pot SSH (con intercambio de claves pública-privada).
    \item Seguminento de incidencias y comentarios de un proyecto; A partir de la interfaz web, los usuarios podrán comentar aspectos del proyecto que vean conveniente discutir. Nos ofrece un servicio de tiqueting para hacer el seguimiento de incidencias u objetivos del proyecto y se puede habilitar un wiki para la documentación que se quiera hacer.
    \item Código del servidor fácilmente accesible remotamente; Trabajando con el servidor proporcionado en nuestra máquina, podemos asegurar la coneión desde el exterior, aislando así un punto de dependencia respecto a un servicio externo. Al tratarse de un servicio de nuestro servidor, podemos activar filtros para limitar el acceso a un rango de redes particular.
\end{itemize}

\section{Webgrafia}
\begin{itemize}
    \item https://git-scm.com/book/es/v2/Git-en-el-Servidor-GitLab 
    \item https://www.freelancer.es/community/articles/github-como-puede-ayudar
    \item https://www.uco.es/aulasoftwarelibre/curso-de-git/cvs/
    \item https://desarrolloweb.com/articulos/introduccion-gitlab.html
    \item https://www.arsys.es/blog/programacion/control-versiones-git/
\end{itemize}



\end{document}
